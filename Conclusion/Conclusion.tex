\section{Conclusion}
This lab covered the spectral analysis of gases, and looking at transmittance and absorption of light through differently-colored solutions. Using mercury to calibrate the spectrometer, we looked at the spectra of Hydrogen and compared them to the expected values from the Balmer series. We also looked at Helium to observe the various spectral lines corresponding to the n=3 to the n=2 transition, and how those interact with the various selection rules of transitions. Finally, we looked at the spectrum of an unknown gas, which we identified as Krypton based on matching our observed wavelengths to the known wavelengths of Krypton.

The second part of the lab covered the transmittance and absorption of light through differently-colored solutions. We looked at the absorbance and transmittance of blue, green, and red dyes, and how that resulted in the color we see when looking at the solution under white light. We then looked at the sepctrum of the yellow dye under excitation with 405nm and 500nm light, and using that to calculate the bandgap to show that the yellow solution could function as a semiconductor.

Thus, we have successfully completed the objectives of the lab, and have a better understanding of the spectral analysis.