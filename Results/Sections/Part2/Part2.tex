
\subsection{Part 2: Hydrogen spectrum}


% Why may the relative intensities of the lines vary for the same atoms in the discharge tube from experiment to experiment?

The relative intensities of the lines may vary for the same atoms in the discharge tube from experiment to experiment due to the fact that the atoms are not in the same energy state.
The atoms in the discharge tube are excited by the electric field, and the energy levels of the atoms are quantized.
This means that the atoms can only have certain energy levels, and the energy levels are not continuous.
The atoms can only emit photons when they transition from one energy level to another, and the energy of the emitted photon is equal to the energy difference between the two energy levels.
The relative intensities of the lines depend on the probability of the atoms being in a certain energy state, and the probability of the atoms being in a certain energy state depends on the temperature of the discharge tube,
the pressure of the discharge tube, and the electric field strength. [add source: https://www.physics.utoronto.ca/~phy293lab/experiments/rydberg.pdf, https://pubmed.ncbi.nlm.nih.gov/1489915/] If the temperature of the discharge tube, the pressure of the discharge tube, or the electric field strength changes, then the relative intensities of the lines will change.
This is why the relative intensities of the lines may vary for the same atoms in the discharge tube from experiment to experiment.

\begin{table}[h]

    \centering
    \begin{adjustbox}{width=1.5\textwidth, center}
        \begin{tabular}{|c|c|c|c|c|}
            \hline
            $n$ & Observed Wavelength (nm) & Theoretical wavelength (nm) & Theoretical line Energy (eV) & Theoretical state energy (eV) \\
            \hline
            3   & $649.50 \pm 5.66$        & 656.39                      & 1.89                         & -1.51                         \\
            4   & $485.70 \pm 4.40$        & 486.21                      & 2.55                         & -0.85                         \\
            5   & $649.50 \pm 5.66$        & 434.12                      & 2.86                         & -0.54                         \\
            \hline
        \end{tabular}
    \end{adjustbox}
    \caption{Observed and theoretical values for the Balmer series of hydrogen.}
    \label{tab:my_label}
\end{table}