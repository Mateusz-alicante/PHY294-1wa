\subsection{Part 3: Calculating energy of quantum states and transmission of helium}

The electronic transmission at the highest intensity of the helium spectrum is expected to occur at around 588 nm \cite{phy293_spectra_nodate}.
Let us first calculate the energy of the quantum states and the energy of the transmission between the states.

For the states $m = 2$ and $n = 3$, we use the formula \ref{eq:Rydberg}, to obtain its energy:

\begin{equation}
    E_2 = -\frac{13.6 \cdot 2^2}{2^2} = -13.6 \, \text{eV} \quad \text{and} \quad E_3 = -\frac{13.6 \cdot 2^2}{3^2} = -6.04  \, \text{eV}
\end{equation}

From here,

\begin{equation}
    \Delta E = E_3 - E_2 = 7.56 \, \text{eV} \quad \text{and} \quad \Delta \lambda = \frac{hc}{\Delta E} = 164.10 \, \text{nm}
\end{equation}

This is clearly not within an acceptable range compared to the value at which the transition does in fact occur.
One reason for this is that when more than one electron is present in the atom, we can no longer make use of the
single electron approximation which equation \ref*{eq:Rydberg} relies on. The complex interactions between the electrons
create additional repulsion, which increases the real wavelengths of the emission spectra.
\begin{table}

    \begin{adjustwidth}{-1cm}{-1cm}
        \begin{adjustbox}{width=1.75\textwidth, center}
            \begin{tabular}{|l|l|l|l|l|l|l|l|l|l|l|l|}
                \hline \begin{tabular}{l}
                           Electron       \\
                           configuration  \\
                           of the initial \\
                           state (upper   \\
                           level)
                       \end{tabular}                                   & \begin{tabular}{l}
                                                                             Electron      \\
                                                                             configuration \\
                                                                             of the final  \\
                                                                             state (lower  \\
                                                                             level)
                                                                         \end{tabular}               & \begin{tabular}{l}
                                                                                                           Reference \\
                                                                                                           intensity \\
                                                                                                           in        \\
                                                                                                           arbitrary \\
                                                                                                           units
                                                                                                       \end{tabular} & \begin{tabular}{l}
                                                                                                                           Reference  \\
                                                                                                                           wavelength \\
                                                                                                                           of the     \\
                                                                                                                           emission   \\
                                                                                                                           line, $\lambda r, \mathrm{~nm}$
                                                                                                                       \end{tabular} & \begin{tabular}{l}
                                                                                                                                           Measured                      \\
                                                                                                                                           wavelength of                 \\
                                                                                                                                           the emission                  \\
                                                                                                                                           line $\lambda_{\text {exp }}$ \\
                                                                                                                                           $+/-$ uncertainty,            \\
                                                                                                                                           $\mathrm{nm}$
                                                                                                                                       \end{tabular} & \begin{tabular}{l}
                                                                                                                                                           Energy  \\
                                                                                                                                                           of the  \\
                                                                                                                                                           quantum \\
                                                                                                                                                           state   \\
                                                                                                                                                           $E_n, \mathrm{eV}$
                                                                                                                                                       \end{tabular} & \begin{tabular}{l}
                                                                                                                                                                           Energy  \\
                                                                                                                                                                           of the  \\
                                                                                                                                                                           quantum \\
                                                                                                                                                                           state   \\
                                                                                                                                                                           $E_m, \mathrm{eV}$
                                                                                                                                                                       \end{tabular} & \begin{tabular}{l}
                                                                                                                                                                                           Selection \\
                                                                                                                                                                                           rule $\Delta n$
                                                                                                                                                                                       \end{tabular} & \begin{tabular}{l}
                                                                                                                                                                                                           Selection \\
                                                                                                                                                                                                           rule $\Delta l$
                                                                                                                                                                                                       \end{tabular} & \begin{tabular}{l}
                                                                                                                                                                                                                           Selection \\
                                                                                                                                                                                                                           rule $\Delta J$
                                                                                                                                                                                                                       \end{tabular}                                                                                          \\
                \hline $1 \mathrm{~s} 2 \mathrm{p}(1)$                      & $1 \mathrm{~s}^2(0)$             & 1000               & 58.43339                                  & -----                            & 21.22                         & -24.57                        & -1                 & -1                 & -1 \\
                \hline \cellcolor{blue!25} $1 \mathrm{~s} 3 \mathrm{~}(1)$  & $1 \mathrm{~s} 2 \mathrm{p}(2)$  & 200                & 706.5190                                  & $713.94 \pm 4.91$                & 1.75                          & -3.60                         & -1                 & 1                  & -1 \\
                \hline \cellcolor{blue!25}  $1 \mathrm{~s} 3 \mathrm{p}(1)$ & $1 \mathrm{~s} 2 \mathrm{~s}(1)$ & 500                & 388.8648                                  & $388.73 \pm 4.30$                & 3.19                          & -4.75                         & -1                 & -1                 & 0  \\
                \hline \cellcolor{blue!25} $1 \mathrm{~s} 3 \mathrm{~d}(3)$ & $1 \mathrm{~s} 2 \mathrm{p}(2)$  & 500                & 587.5621                                  & $588.24 \pm 5.52$                & 2.11                          & -3.60                         & -1                 & -1                 & -1 \\
                \hline \cellcolor{blue!25} $1 \mathrm{~s} 3 \mathrm{~d}(2)$ & $1 \mathrm{~s} 2 \mathrm{p}(1)$  & 100                & 667.8151                                  & $673.25 \pm 5.52$                & 1.86                          & -3.35                         & -1                 & -1                 & -1 \\
                \hline \cellcolor{blue!25} $1 \mathrm{~s} 3 \mathrm{p}(1)$  & $1 \mathrm{~s} 2 \mathrm{~s}(0)$ & 100                & 501.56783                                 & $501.19 \pm 3.68$                & 2.47                          & -3.95                         & -1                 & -1                 & -1 \\
                \hline $1 \mathrm{~s} 4 \mathrm{~d}(1)$                     & $1 \mathrm{~s} 2 \mathrm{p}(2)$  & 200                & 447.14802                                 & $447.26 \pm 3.07$                & 2.77                          & -3.60                         & -2                 & -1                 & -1 \\
                \hline
            \end{tabular}

        \end{adjustbox}

    \end{adjustwidth}
    \caption{Observed and theoretical values for the Helium atom.}
    \label{tab:heliumSpectra}
\end{table}

In table \ref{tab:heliumSpectra}, we can see different parameters like wavelength and energy levels of the Helium atom.

The 5 transitions corresponding to jumps from $n = 3$ to $m = 2$ are highlighted in {\colorbox{blue!25} {blue}}. These could not be observed for the Hydrogen orbitals since the single electron means that the energy levels are all degenerate, meaning transitioning from 3s to 2s would be equivalent to transitioning from 3p to 2s. For Helium, however, different angular momentum states have different energies due to the different radial distributions of the wavefunctions.

The selection rules show that $\Delta n < 0$,  $\Delta l = \pm 1$, and $\Delta J = 0, \pm 1$. 
